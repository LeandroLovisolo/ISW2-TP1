\documentclass[a4paper, 10pt, twoside]{article}

\usepackage[top=1in, bottom=1in, left=1in, right=1in]{geometry}
\usepackage[utf8]{inputenc}
\usepackage[spanish, es-ucroman, es-noquoting]{babel}
\usepackage{setspace}
\usepackage{fancyhdr}
\usepackage{lastpage}
\usepackage{amsmath}
\usepackage{amsfonts}
\usepackage{amsthm}
\usepackage{verbatim}
\usepackage{fancyvrb}
\usepackage{graphicx}
\usepackage{float}
\usepackage{enumitem} % Provee macro \setlist
\usepackage{tabularx}
\usepackage{multirow}
\usepackage{hyperref}
\usepackage{xspace}
\usepackage{qtree}
\usepackage[toc, page]{appendix}


%%%%%%%%%% Constantes - Inicio %%%%%%%%%%
\newcommand{\titulo}{Trabajo Práctico 1}
\newcommand{\materia}{Ingeniería de Software II}
\newcommand{\integrantes}{Izcovich · Lovisolo · Petaccio · Vita}
\newcommand{\cuatrimestre}{Primer Cuatrimestre de 2015}
%%%%%%%%%% Constantes - Fin %%%%%%%%%%


%%%%%%%%%% Configuración de Fancyhdr - Inicio %%%%%%%%%%
\pagestyle{fancy}
\thispagestyle{fancy}
\lhead{\titulo\ · \materia}
\rhead{\integrantes}
\renewcommand{\footrulewidth}{0.4pt}
\cfoot{\thepage /\pageref{LastPage}}

\fancypagestyle{caratula} {
   \fancyhf{}
   \cfoot{\thepage /\pageref{LastPage}}
   \renewcommand{\headrulewidth}{0pt}
   \renewcommand{\footrulewidth}{0pt}
}
%%%%%%%%%% Configuración de Fancyhdr - Fin %%%%%%%%%%


%%%%%%%%%% Miscelánea - Inicio %%%%%%%%%%
% Evita que el documento se estire verticalmente para ocupar el espacio vacío
% en cada página.
\raggedbottom

% Separación entre párrafos.
\setlength{\parskip}{0.5em}

% Separación entre elementos de listas.
\setlist{itemsep=0.5em}

% Asigna la traducción de la palabra 'Appendices'.
\renewcommand{\appendixtocname}{Apéndices}
\renewcommand{\appendixpagename}{Apéndices}

\newcommand{\grafico}[1]{
  \begin{center}
    \includegraphics[height=10cm]{#1}
  \end{center}
}


%%%%%%%%%% Miscelánea - Fin %%%%%%%%%%


%%%%%%%%%% User stories y tareas - Inicio %%%%%%%%%%
% Entorno dentro del cual se declaran las stories con el macro \story.
\newenvironment{stories}{
  \begin{itemize}
}{
  \end{itemize}
}

% Uso: \story{id}{rol}{historia}{criterio de aceptación}.
\newcommand{\story}[4]{
  \item
  \textbf{ID #1:} Como \emph{#2} quiero \emph{#3} para \emph{#4}.
}

% Uso: \storyid{id}{rol}{historia}{criterio de aceptación}.
\newcommand{\storyid}[4]{
  \item
  \textbf{ID #1:} Como \emph{#2} quiero \emph{#3} para \emph{#4}.
}

% Entorno dentro del cual se declaran las tareas de un story con el macro \task.
\newenvironment{tasks}{
  Tareas:
  \begin{enumerate}
}{
  \end{enumerate}
}

% Uso: \task{descripción de la story}
\newcommand{\task}[1] {
  \item #1.\\
  \textbf{Horas estimadas:}
}

% Entorno dentro del cual se declaran las descripciones de un story con el macro \task.
\newenvironment{details}{
  Descripción:
  \begin{itemize}
}{
  \end{itemize}
}

% Uso: \detail{descripción de la story}
\newcommand{\detail}[1] {
  \item #1.
}

% Entorno dentro del cual se declaran los criterios de aceptación de un story con el macro \criteria.
\newenvironment{criterios}{
  Criterio de aceptación:
  \begin{itemize}
}{
  \end{itemize}
}

% Uso: \detail{criterio de aceptación}
\newcommand{\criteria}[1] {
  \item #1
}

% Uso: \storydetails{id}{descripción}{detalles}{criterios de aceptación}{tareas}
\newcommand{\storydetails}[5]{
  \noindent
  \textbf{ID #1: #2}

  \noindent
  Detalles:
  \begin{itemize}
    #3
  \end{itemize}

  \noindent
  Criterio de aceptación:
  \begin{itemize}
    #4
  \end{itemize}

  \noindent
  Tareas:
  \begin{itemize}
    #5
  \end{itemize}
}
%%%%%%%%%% User stories y tareas - Fin %%%%%%%%%%


\begin{document}


%%%%%%%%%%%%%%%%%%%%%%%%%%%%%%%%%%%%%%%%%%%%%%%%%%%%%%%%%%%%%%%%%%%%%%%%%%%%%%%
%% Carátula                                                                  %%
%%%%%%%%%%%%%%%%%%%%%%%%%%%%%%%%%%%%%%%%%%%%%%%%%%%%%%%%%%%%%%%%%%%%%%%%%%%%%%%


\thispagestyle{caratula}

\begin{center}

\includegraphics[height=2cm]{DC.png} 
\hfill
\includegraphics[height=2cm]{UBA.jpg} 

\vspace{2cm}

Departamento de Computación,\\
Facultad de Ciencias Exactas y Naturales,\\
Universidad de Buenos Aires

\vspace{4cm}

\begin{Huge}
\titulo
\end{Huge}

\vspace{0.5cm}

\begin{Large}
\materia
\end{Large}

\vspace{1cm}

\cuatrimestre

\vspace{4cm}

\begin{tabular}{|c|c|c|}
\hline
Apellido y Nombre & LU & E-mail\\
\hline
Izcovich, Sabrina      & 550/11 & sizcovich@gmail.com\\
Lovisolo, Leandro      & 645/11 & leandro@leandro.me\\
Petaccio, Lautaro José & 443/11 & lausuper@gmail.com\\
Vita, Sebastián        & 149/11 & sebastian\_vita@yahoo.com.ar\\
\hline
\end{tabular}

\end{center}

\newpage

\tableofcontents

\newpage


%%%%%%%%%%%%%%%%%%%%%%%%%%%%%%%%%%%%%%%%%%%%%%%%%%%%%%%%%%%%%%%%%%%%%%%%%%%%%%%
%% Introducción                                                              %%
%%%%%%%%%%%%%%%%%%%%%%%%%%%%%%%%%%%%%%%%%%%%%%%%%%%%%%%%%%%%%%%%%%%%%%%%%%%%%%%

\section{Introducción}

En este trabajo se desarrolla una aplicación de envío de avisos, recordatorios y otros tipos de mensajes por medio de SMS a los padres y alumnos de un colegio primario.

La aplicación se especifica y estima utilizando la metodología ágil Scrum, se realiza un diseño bajo el paradigma orientado a objetos respetando un conjunto de reglas de buenas prácticas de diseño y, finalmente, se la implementa en el lenguaje Python.

Para más información, referirse al enunciado de este trabajo práctico\footnote{http://cor.to/enunciadoISW2}.

\subsection{Roles de usuario}

Por brevedad, se utilizan los siguientes roles en las user stories a lo largo de este trabajo:

\begin{description}
  \item[Director] para referirse al director, vicedirector y otros miembros de la dirección del colegio.

  \item[Secretario] para referirse a los miembros de la secretaría del colegio.

  \item[Maestro] para referirse a cualquier maestro del colegio.
\end{description}

\newpage
%%%%%%%%%%%%%%%%%%%%%%%%%%%%%%%%%%%%%%%%%%%%%%%%%%%%%%%%%%%%%%%%%%%%%%%%%%%%%%%
%% Product backlog                                                           %%
%%%%%%%%%%%%%%%%%%%%%%%%%%%%%%%%%%%%%%%%%%%%%%%%%%%%%%%%%%%%%%%%%%%%%%%%%%%%%%%

\section{Product backlog}

Presentamos, a continuación, las user stories en el product backlog, agrupadas según el tipo de tarea que describen.


\subsection{Gestión de alumnos, padres, cursos y maestros}

\begin{stories}
  \story{1}{director}
        {ingresar los datos de un alumno}
        {inscribirlo en el colegio} 

  \story{2}{director}
        {ver los datos de un alumno}
        {actualizarlos o dar de baja al alumno} 

  \story{3}{director}
        {ingresar los datos de un padre}
        {darlo de alta en el sistema} 

  \story{4}{director}
        {ver los datos de un padre}
        {actualizarlos o dar de baja al padre} 

  \story{5}{director}
        {seleccionar un padre}
        {asignarlo como padre de un alumno inscripto} 

  \story{6}{director}
        {ingresar el grado, división, turno y aula de un curso}
        {darlo de alta en el sistema} 

  \story{7}{director}
        {ver los datos de un curso}
        {actualizarlos o dar de baja el curso} 

  \story{8}{director}
        {ingresar los datos de un maestro}
        {darlo de alta en el sistema} 

  \story{9}{director}
        {ver los datos de un maestro}
        {actualizarlos o dar de baja al maestro} 

  \story{10}{director}
        {seleccionar un maestro}
        {asignárselo a un curso} 

  \story{11}{director}
        {seleccionar un conjunto de alumnos}
        {asignarles un curso} 
\end{stories}

\begin{center}
\begin{tabular}{|c|c|c|c|}
\hline
ID & Story points & Business value\\
\hline
1 & 3 & 3\\
2 & 2 & 3\\
3 & 3 & 3\\
4 & 2 & 3\\
5 & 2 & 3\\
6 & 3 & 3\\
7 & 2 & 3\\
8 & 3 & 3\\
9 & 2 & 3\\
10 & 2 & 3\\
11 & 2 & 3\\
\hline
\end{tabular}
\end{center}

\subsection{Gestión de eventos}

\begin{stories}
  \story{12}{director, secretario o maestro}
        {ingresar la fecha y descripción de un nuevo evento}
        {darlo de alta en el sistema} 

  \story{13}{director, secretario o maestro}
        {ver los datos de un evento}
        {modificarlos o dar de baja el evento} 
\end{stories}

\begin{center}
\begin{tabular}{|c|c|c|c|}
\hline
ID & Story points & Business value\\
\hline
12 & 3 & 1\\
13 & 2 & 3\\
\hline
\end{tabular}
\end{center}

\subsection{Gestión de campañas}

\begin{stories}
  \story{14}{director, secretario o maestro}
        {ingresar el título de una nueva campaña y seleccionar un evento existente}
        {crear una campaña asociada al evento seleccionado} 

  \story{15}{maestro}
        {seleccionar un subconjunto de mis alumnos}
        {asignarlos como destinatarios de una campaña creada por mí} 

  \story{16}{director o secretario}
        {seleccionar un subconjunto de los alumnos del colegio}
        {asignarlos como destinatarios de una campaña creada por mí} 

  \story{17}{director, secretario o maestro}
        {ingresar un mensaje, fecha, hora y tipo}
        {agregar un nuevo mensaje a una campaña creada por mí} 

  \story{18}{director, secretario o maestro}
        {ver los mensajes de una campaña creada por mí} 
        {modificarlos o eliminarlos}

  \story{19}{director, secretario o maestro}
        {que se envíen automáticamente y en el momento correcto los mensajes de una campaña creada por mí}
        {que sean recibidos en el momento adecuado por sus destinatarios} 

  \story{20}{director, secretario o maestro}
        {ingresar la eficacia de una campaña luego de su terminación}
        {compararla con otras campañas similares} 
\end{stories}

\begin{center}
\begin{tabular}{|c|c|c|c|}
\hline
ID & Story points & Business value\\
\hline
14 & 3 & 1\\
15 & 2 & 1\\
16 & 2 & 1\\
17 & 2 & 1\\
18 & 5 & 1\\
19 & 8 & 1\\
20 & 5 & 1\\
\hline
\end{tabular}
\end{center}

%%%%%%%%%%%%%%%%%%%%%%%%%%%%%%%%%%%%%%%%%%%%%%%%%%%%%%%%%%%%%%%%%%%%%%%%%%%%%%%
%% Sprint backlog                                                            %%
%%%%%%%%%%%%%%%%%%%%%%%%%%%%%%%%%%%%%%%%%%%%%%%%%%%%%%%%%%%%%%%%%%%%%%%%%%%%%%%

\newpage
\section{Sprint backlog}

Las user stories a continuación se toman del product backlog y corresponden al sprint luego del cual se realiza la demo mencionada en el enunciado.

Recordamos los requerimientos de la demo: \emph{``Se presentará una demo requiriendo cierta funcionalidad pedida desde los responsables del proyecto. La misma deberá incluir la posibilidad de definir una campaña para un cierto evento, simular su seguimiento y envío de mensajes, y su posterior evaluación''}.


\subsection{Descripción de user stories, criterios de aceptación y tareas}

\begin{stories}
  \storyid{14}
          {director, secretario o maestro}
          {ingresar el título de una nueva campaña y seleccionar un evento existente}
          {crear una campaña asociada al evento seleccionado}

   \begin{criterios}
     \criteria{El formulario no se envía si el título ingresado es menor a tres caracteres o si no se selecciona un evento.}
     \criteria{Luego de enviarse el formulario, se crea una campaña en la base de datos con la información ingresada.}
  \end{criterios}

  \begin{tasks}
    \task{Crear interfaz de creación de campaña} 4
    \task{Crear tabla de campañas en la base de datos} 1
    \task{Crear código de persistencia de nuevas campañas} 1
  \end{tasks}


  \storyid{15}
          {maestro}
          {seleccionar un subconjunto de mis alumnos}
          {asignarlos como destinatarios de una campaña creada por mí}

   \begin{criterios}
    \criteria{El maestro sólo debe poder seleccionar alumnos de los cursos en los que él es maestro.}
    \criteria{El formulario no se envía si no se selecciona al menos un alumno.}
    \criteria{Luego de enviado el formulario, se reemplaza la lista de destinatarios de la campaña en la base de datos con los alumnos seleccionados.}
  \end{criterios}

  \begin{tasks}
    \task{Modificar componente de selección de destinatarios para sólo permitir seleccionar alumnos del maestro y/o sus padres cuando el usuario actual tenga el rol de maestro} 2
  \end{tasks}


  \storyid{16}
          {director o secretario}
          {seleccionar un subconjunto de los alumnos del colegio}
          {asignarlos como destinatarios de una campaña creada por mí}

  \begin{criterios}
    \criteria{El formulario no se envía si no se selecciona al menos un alumno.}
    \criteria{Luego de enviado el formulario, se reemplaza la lista de destinatarios de la campaña en la base de datos con los alumnos seleccionados.}
  \end{criterios}

  \begin{tasks}
    \task{Agregar componente en el formulario de creación de campaña para seleccionar los destinatarios de la misma} 0.5
    \task{Crear tablas de relación entre las campañas y los distintos tipos de destinatarios} 1
    \task{Crear código de persistencia para la lista de destinatarios de la campaña} 1
  \end{tasks}


  \storyid{17}{director, secretario o maestro}
              {ingresar un mensaje, fecha, hora y tipo}
              {agregar un nuevo mensaje a una campaña creada por mí}

  \begin{criterios}
    \criteria{El formulario no se envía si no se completan todos los campos, o si la fecha/hora ingresada es inválida, o si ya hay otro mensaje con la misma fecha/hora ingresada.}
    \criteria{Luego de enviado el formulario, se crea en la base de datos un nuevo mensaje asociado a la campaña desde la que se llegó al formulario, con la información ingresada en el formulario.}
  \end{criterios}

  \begin{tasks}
    \task{Agregar componente tabular en el formulario de creación de campaña que permita ingresar uno a uno los mensajes y visualizar los ya ingresados} 1
    \task{Crear tabla de mensajes en la base de datos} 1
    \task{Crear codigo de persistencia de los mensajes ingresados} 1
  \end{tasks}


  \storyid{18}
          {director, secretario o maestro}
          {ver los mensajes de una campaña creada por mí}
          {modificarlos o eliminarlos}

   \begin{criterios}
    \criteria{En caso de modificar un mensaje, el formulario debe precargarse con los datos del mensaje original, no se envía si no se completan todos los campos, o si la fecha/hora ingresada es inválida, o si ya hay otro mensaje con la misma fecha/hora ingresada.}
    \criteria{Luego de enviado el formulario, se actualiza el mensaje en la base de datos con la información ingresada en el formulario.}
    \criteria{En caso de eliminar un mensaje, debe ofrecerse una confirmación antes de realizar la operación.}
    \criteria{Luego de finalizada la operación, debe eliminarse el mensaje en la base de datos.}
  \end{criterios}

  \begin{tasks}
    \task{Crear una interfaz que liste todas las campañas creadas por el usuario actual} 3
    \task{Crear interacción que muestre el formulario de edición de la campaña cuando se la selecciona en la lista anterior} 2
    \task{Modificar formulario de edición de campaña y sus componentes para que tenga precargados los datos de una campaña existente y permita modificarla} 0.5
    \task{Modificar código de persistencia de campañas para que permita guardar cambios en una campaña existente} 0.5
  \end{tasks}


  \storyid{19}{director, secretario o maestro}
              {que se envíen automáticamente y en el momento correcto los mensajes de una campaña creada por mí}
              {que sean recibidos en el momento adecuado por sus destinatarios}

   \begin{criterios}
    \criteria{El envío de mensajes debe realizarse automáticamente a la hora y destinatarios estipulados.}
    \criteria{Los mensajes enviados deben guardarse en una tabla en la base de datos donde quedan registrados los envíos simulados de SMS.}
    \criteria{Todo mensaje cuyo envío es hsimulado debe ser marcado como enviado en la base de datos.}
  \end{criterios}

  \begin{tasks}
    \task{Crear una API de envío de SMS que simule el envío guardando los mensajes enviados en una tabla en la base de datos} 8
    \task{Crear un servicio que periódicamente envíe los mensajes cuyo horario fue alcanzado utilizando el API de envío de SMS} 6
  \end{tasks}


  \storyid{20}{director, secretario o maestro}
              {ingresar la eficacia de una campaña luego de su terminación}
              {compararla con otras campañas similares}

  \begin{criterios}
    \criteria{El formulario no se envía si no se completan todos los campos, o si el valor de la eficacia medida no es un valor numérico.}
    \criteria{Luego de enviarse el formulario, se actualiza la campaña en la base de datos con la información de eficacia ingresada.}
  \end{criterios}

  \begin{tasks}
    \task{Agregar un campo textual en el formulario de edición de campaña que permita especificar un criterio de eficacia en lenguaje natural} 1
    \task{Agregar un campo numérico en el formulario de edición de campaña que permita ingresar el valor de eficacia medido} 1
    \task{Agregar campos en la tabla de campaña para almacenar el criterio de eficacia y la eficacia medida} 4
    \task{Modificar código de persistencia de campaña para guardar el criterio de eficacia y la eficacia medida} 2
  \end{tasks}
\end{stories}

\end{document}
