
\documentclass[a4paper, 10pt, twoside]{article}

\usepackage[top=1in, bottom=1in, left=1in, right=1in]{geometry}
\usepackage[utf8]{inputenc}
\usepackage[spanish, es-ucroman, es-noquoting]{babel}
\usepackage{setspace}
\usepackage{fancyhdr}
\usepackage{lastpage}
\usepackage{amsmath}
\usepackage{amsfonts}
\usepackage{amsthm}
\usepackage{verbatim}
\usepackage{fancyvrb}
\usepackage{graphicx}
\usepackage{float}
\usepackage{enumitem} % Provee macro \setlist
\usepackage{tabularx}
\usepackage{multirow}
\usepackage{hyperref}
\usepackage{xspace}
\usepackage{qtree}
\usepackage[toc, page]{appendix}


%%%%%%%%%% Constantes - Inicio %%%%%%%%%%
\newcommand{\titulo}{Trabajo Práctico 1}
\newcommand{\materia}{Ingeniería de Software II}
\newcommand{\integrantes}{Izcovich · Lovisolo · Petaccio · Vita}
\newcommand{\cuatrimestre}{Primer Cuatrimestre de 2015}
%%%%%%%%%% Constantes - Fin %%%%%%%%%%


%%%%%%%%%% Configuración de Fancyhdr - Inicio %%%%%%%%%%
\pagestyle{fancy}
\thispagestyle{fancy}
\lhead{\titulo\ · \materia}
\rhead{\integrantes}
\renewcommand{\footrulewidth}{0.4pt}
\cfoot{\thepage /\pageref{LastPage}}

\fancypagestyle{caratula} {
   \fancyhf{}
   \cfoot{\thepage /\pageref{LastPage}}
   \renewcommand{\headrulewidth}{0pt}
   \renewcommand{\footrulewidth}{0pt}
}
%%%%%%%%%% Configuración de Fancyhdr - Fin %%%%%%%%%%


%%%%%%%%%% Miscelánea - Inicio %%%%%%%%%%
% Evita que el documento se estire verticalmente para ocupar el espacio vacío
% en cada página.
\raggedbottom

% Separación entre párrafos.
\setlength{\parskip}{0.5em}

% Separación entre elementos de listas.
\setlist{itemsep=0.5em}

% Asigna la traducción de la palabra 'Appendices'.
\renewcommand{\appendixtocname}{Apéndices}
\renewcommand{\appendixpagename}{Apéndices}

\newcommand{\grafico}[1]{
  \begin{center}
    \includegraphics[height=10cm]{#1}
  \end{center}
}


%%%%%%%%%% Miscelánea - Fin %%%%%%%%%%


%%%%%%%%%% User stories y tareas - Inicio %%%%%%%%%%
% Entorno dentro del cual se declaran las stories con el macro \story.
\newenvironment{stories}{
  \begin{itemize}
}{
  \end{itemize}
}

% Uso: \story{rol}{historia}{criterio de aceptación}.
\newcommand{\story}[3]{
  \item
  Como \emph{#1} quiero \emph{#2} para \emph{#3}.
}

% Entorno dentro del cual se declaran las tareas de un story con el macro \task.
\newenvironment{tasks}{
  Tareas:
  \begin{itemize}
}{
  \end{itemize}
}

% Uso: \task{descripción de la tarea}
\newcommand{\task}[1] {
  \item #1.
}
%%%%%%%%%% User stories y tareas - Fin %%%%%%%%%%


\begin{document}


%%%%%%%%%%%%%%%%%%%%%%%%%%%%%%%%%%%%%%%%%%%%%%%%%%%%%%%%%%%%%%%%%%%%%%%%%%%%%%%
%% Carátula                                                                  %%
%%%%%%%%%%%%%%%%%%%%%%%%%%%%%%%%%%%%%%%%%%%%%%%%%%%%%%%%%%%%%%%%%%%%%%%%%%%%%%%


\thispagestyle{caratula}

\begin{center}

\includegraphics[height=2cm]{DC.png} 
\hfill
\includegraphics[height=2cm]{UBA.jpg} 

\vspace{2cm}

Departamento de Computación,\\
Facultad de Ciencias Exactas y Naturales,\\
Universidad de Buenos Aires

\vspace{4cm}

\begin{Huge}
\titulo
\end{Huge}

\vspace{0.5cm}

\begin{Large}
\materia
\end{Large}

\vspace{1cm}

\cuatrimestre

\vspace{4cm}

\begin{tabular}{|c|c|c|}
\hline
Apellido y Nombre & LU & E-mail\\
\hline
Izcovich, Sabrina      & 550/11 & sizcovich@gmail.com\\
Lovisolo, Leandro      & 645/11 & leandro@leandro.me\\
Petaccio, Lautaro José & 443/11 & lausuper@gmail.com\\
Vita, Sebastián        & 149/11 & sebastian\_vita@yahoo.com.ar\\
\hline
\end{tabular}

\end{center}

\newpage


%%%%%%%%%%%%%%%%%%%%%%%%%%%%%%%%%%%%%%%%%%%%%%%%%%%%%%%%%%%%%%%%%%%%%%%%%%%%%%%
%% Introducción                                                              %%
%%%%%%%%%%%%%%%%%%%%%%%%%%%%%%%%%%%%%%%%%%%%%%%%%%%%%%%%%%%%%%%%%%%%%%%%%%%%%%%

\section{Introducción}

En este trabajo se desarrolla una aplicación de envío de avisos, recordatorios y otros tipos de mensajes por medio de SMS a los padres y alumnos de un colegio primario.

La aplicación se especifica y estima utilizando la metodología ágil Scrum, se realiza un diseño bajo el paradigma orientado a objetos respetando un conjunto de reglas de buenas prácticas de diseño, y finalmente se la implementa en el lenguaje Python.

Para más información, referirise al enunciado de este trabajo práctico.


\subsection{Roles de usuario}

Por brevedad, se utilizan los siguientes roles en las user stories a lo largo de este trabajo:

\begin{description}
  \item[Director] para referirse al director, vicedirector y otros miembros de la dirección del colegio.

  \item[Secretario] para referirse a los miembros de la secretaría del colegio.

  \item[Maestro] para referirse a cualquier maestro del colegio.
\end{description}


%%%%%%%%%%%%%%%%%%%%%%%%%%%%%%%%%%%%%%%%%%%%%%%%%%%%%%%%%%%%%%%%%%%%%%%%%%%%%%%
%% Product backlog                                                           %%
%%%%%%%%%%%%%%%%%%%%%%%%%%%%%%%%%%%%%%%%%%%%%%%%%%%%%%%%%%%%%%%%%%%%%%%%%%%%%%%

\section{Product backlog}

Presentamos a continuación las user stories en el product backlog, agrupadas según el tipo de tarea que describen.


\subsection{Gestión de alumnos, padres, cursos y maestros}

\begin{stories}
  \story{director}
        {ingresar los datos de un alumno}
        {inscribirlo en el colegio}

  \story{director}
        {ver los datos de un alumno}
        {actualizarlos o dar de baja al alumno}

  \story{director}
        {ingresar los datos de un padre}
        {darlo de alta en el sistema}

  \story{director}
        {ver los datos de un padre}
        {actualizarlos o dar de baja al padre}
       
  \story{director}
        {seleccionar un padre}
        {asignarlo como padre de un alumno inscripto}

  \story{director}
        {ingresar el grado, división, turno y aula de un curso}
        {darlo de alta en el sistema}

  \story{director}
        {ver los datos de un curso}
        {actualizarlos o dar de baja el curso}

  \story{director}
        {ingresar los datos de un maestro}
        {darlo de alta en el sistema}

  \story{director}
        {ver los datos de un maestro}
        {actualizarlos o dar de baja al maestro}

  \story{director}
        {seleccionar un maestro}
        {asignárselo a un curso}

  \story{director}
        {seleccionar un conjunto de alumnos}
        {asignarles un curso}
\end{stories}


\subsection{Gestión de eventos}

\begin{stories}
  \story{director, secretario ó maestro}
        {ingresar la fecha y descripción de un evento}
        {darlo de alta en el sistema}

  \story{director, secretario ó maestro}
        {ver los datos de un evento}
        {modificarlos o dar de baja el evento}
\end{stories}


\subsection{Gestión de campañas}

\begin{stories}
  \story{director, secretario ó maestro}
        {seleccionar un evento}
        {crear una campaña asociada al evento seleccionado}

  \story{maestro}
        {seleccionar un subconjunto de mis alumnos}
        {asignarlos como destinatarios de una campaña que yo creé}

  \story{director ó secretario}
        {seleccionar un subconjunto de los alumnos del colegio}
        {asignarlos como destinatarios de una campaña que yo creé}

  \story{director, secretario ó maestro}
        {ingresar un mensaje, fecha, hora y tipo}
        {agregar un nuevo mensaje a una campaña que yo creé}

  \story{director, secretario ó maestro}
        {ver los mensajes de una campaña que yo creé}
        {modificarlos o eliminarlos}

  \story{director, secretario ó maestro}
        {que se envíen automáticamente y en el momento correcto los mensajes de una campaña que yo creé}
        {incrementar la eficacia de un evento}

  \story{director, secretario ó maestro}
        {ingresar la eficacia de una campaña luego de su terminación}
        {compararla con otras campañas similares}
\end{stories}


%%%%%%%%%%%%%%%%%%%%%%%%%%%%%%%%%%%%%%%%%%%%%%%%%%%%%%%%%%%%%%%%%%%%%%%%%%%%%%%
%% Sprint backlog                                                            %%
%%%%%%%%%%%%%%%%%%%%%%%%%%%%%%%%%%%%%%%%%%%%%%%%%%%%%%%%%%%%%%%%%%%%%%%%%%%%%%%


\section{Sprint backlog}

Las user stories a continuación se toman del product backlog y corresponden al sprint luego del cual se realiza la demo mencionada en el enunciado.

Recordamos los requerimientos de la demo: \emph{``Se presentará una demo requiriendo cierta funcionalidad pedida desde los responsables del proyecto. La misma deberá incluir la posibilidad de definir una campaña para un cierto evento, simular su seguimiento y envío de mensajes, y su posterior evaluación''}.

\begin{stories}
  \story{director, secretario ó maestro}
        {ingresar la fecha y descripción de un nuevo evento}
        {darlo de alta en el sistema}

  \begin{tasks}
    \task{Crear interfaz de creación de eventos}
    \task{Crear tabla de eventos en la base de datos}
    \task{Crear código de persistencia de nuevos eventos}
  \end{tasks}

  \story{director, secretario ó maestro}
        {ingresar el título de una nueva campaña y seleccionar un evento existente}
        {crear una campaña asociada al evento seleccionado}

  \begin{tasks}
    \task{Crear interfaz de creación de campaña}
    \task{Crear tabla de campañas en la base de datos}
    \task{Crear código de persistencia de nuevas campañas}
  \end{tasks}

  \story{maestro}
        {seleccionar un subconjunto de mis alumnos}
        {asignarlos como destinatarios de una campaña que yo creé}

  \begin{tasks}
    \task{mostrar todas las campañas que creé}
    \task{seleccionar la campaña deseada}
    \task{mostrar todos los alumnos que tengo a cargo}
    \task{seleccionar un subconjunto de alumnos}
    \task{cargar alumnos a la campaña}
  \end{tasks}

  \story{director ó secretario}
        {seleccionar un subconjunto de los alumnos del colegio}
        {asignarlos como destinatarios de una campaña que yo creé}

  \begin{tasks}
    \task{mostrar todas las campañas que creé}
    \task{seleccionar la campaña deseada}
    \task{mostrar todos los alumnos del colegio}
    \task{seleccionar un subconjunto de alumnos}
    \task{cargar alumnos a la campaña}
  \end{tasks}

  \story{director, secretario ó maestro}
        {ingresar un mensaje, fecha, hora y tipo}
        {agregar un nuevo mensaje a una campaña que yo creé}

  \begin{tasks}
    \task{mostrar todas las campañas que creé}
    \task{seleccionar la campaña deseada}
    \task{ingresar el mensaje}
    \task{ingresar fecha y hora en la que se desea enviar el mensaje}
    \task{ingresar si el mensaje es del tipo anuncio/recordatorio/asignación/mensaje de aliento}
    \task{cargar el mensaje en el sistema}
  \end{tasks}

  \story{director, secretario ó maestro}
        {ver los mensajes de una campaña que yo creé}
        {modificarlos o eliminarlos}

  \story{director, secretario ó maestro}
        {que se envíen automáticamente y en el momento correcto los mensajes de una campaña que yo creé}
        {incrementar la eficacia de un evento}

  \story{director, secretario ó maestro}
        {ingresar la eficacia de una campaña luego de su terminación}
        {compararla con otras campañas similares}
\end{stories}

\end{document}
