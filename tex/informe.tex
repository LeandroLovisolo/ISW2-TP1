\documentclass[a4paper, 10pt, twoside]{article}

\usepackage[top=1in, bottom=1in, left=1in, right=1in]{geometry}
\usepackage[utf8]{inputenc}
\usepackage[spanish, es-ucroman, es-noquoting]{babel}
\usepackage{setspace}
\usepackage{fancyhdr}
\usepackage{lastpage}
\usepackage{amsmath}
\usepackage{amsfonts}
\usepackage{amsthm}
\usepackage{verbatim}
\usepackage{fancyvrb}
\usepackage{graphicx}
\usepackage{float}
\usepackage{enumitem} % Provee macro \setlist
\usepackage{tabularx}
\usepackage{multirow}
\usepackage{hyperref}
\usepackage{xspace}
\usepackage{qtree}
\usepackage[toc, page]{appendix}


%%%%%%%%%% Constantes - Inicio %%%%%%%%%%
\newcommand{\titulo}{Trabajo Práctico 1}
\newcommand{\materia}{Ingeniería de Software II}
\newcommand{\integrantes}{Izcovich · Lovisolo · Petaccio · Vita}
\newcommand{\cuatrimestre}{Primer Cuatrimestre de 2015}
%%%%%%%%%% Constantes - Fin %%%%%%%%%%


%%%%%%%%%% Configuración de Fancyhdr - Inicio %%%%%%%%%%
\pagestyle{fancy}
\thispagestyle{fancy}
\lhead{\titulo\ · \materia}
\rhead{\integrantes}
\renewcommand{\footrulewidth}{0.4pt}
\cfoot{\thepage /\pageref{LastPage}}

\fancypagestyle{caratula} {
   \fancyhf{}
   \cfoot{\thepage /\pageref{LastPage}}
   \renewcommand{\headrulewidth}{0pt}
   \renewcommand{\footrulewidth}{0pt}
}
%%%%%%%%%% Configuración de Fancyhdr - Fin %%%%%%%%%%


%%%%%%%%%% Miscelánea - Inicio %%%%%%%%%%
% Evita que el documento se estire verticalmente para ocupar el espacio vacío
% en cada página.
\raggedbottom

% Separación entre párrafos.
\setlength{\parskip}{0.5em}

% Separación entre elementos de listas.
\setlist{itemsep=0.5em}

% Asigna la traducción de la palabra 'Appendices'.
\renewcommand{\appendixtocname}{Apéndices}
\renewcommand{\appendixpagename}{Apéndices}

\newcommand{\grafico}[1]{
  \begin{center}
    \includegraphics[height=10cm]{#1}
  \end{center}
}


%%%%%%%%%% Miscelánea - Fin %%%%%%%%%%


%%%%%%%%%% User stories y tareas - Inicio %%%%%%%%%%
% Entorno dentro del cual se declaran las stories con el macro \story.
\newenvironment{stories}{
  \begin{itemize}
}{
  \end{itemize}
}

% Uso: \story{rol}{historia}{criterio de aceptación}.
\newcommand{\story}[3]{
  \item
  Como \emph{#1} quiero \emph{#2} para \emph{#3}.
}

% Uso: \storyid{id}{rol}{historia}{criterio de aceptación}.
\newcommand{\storyid}[4]{
  \item
  \textbf{ID #1:} Como \emph{#2} quiero \emph{#3} para \emph{#4}.
}

% Entorno dentro del cual se declaran las tareas de un story con el macro \task.
\newenvironment{tasks}{
  Tareas:
  \begin{enumerate}
}{
  \end{enumerate}
}

% Uso: \task{descripción de la story}
\newcommand{\task}[1] {
  \item #1.\\
  \textbf{Horas estimadas:}
}

% Entorno dentro del cual se declaran las descripciones de un story con el macro \task.
\newenvironment{details}{
  Descripción:
  \begin{itemize}
}{
  \end{itemize}
}

% Uso: \detail{descripción de la story}
\newcommand{\detail}[1] {
  \item #1.
}

% Entorno dentro del cual se declaran los criterios de aceptación de un story con el macro \criteria.
\newenvironment{criterios}{
  Criterio de aceptación:
  \begin{itemize}
}{
  \end{itemize}
}

% Uso: \detail{criterio de aceptación}
\newcommand{\criteria}[1] {
  \item #1
}

% Uso: \storydetails{id}{descripción}{detalles}{criterios de aceptación}{tareas}
\newcommand{\storydetails}[5]{
  \noindent
  \textbf{ID #1: #2}

  \noindent
  Detalles:
  \begin{itemize}
    #3
  \end{itemize}

  \noindent
  Criterio de aceptación:
  \begin{itemize}
    #4
  \end{itemize}

  \noindent
  Tareas:
  \begin{itemize}
    #5
  \end{itemize}
}
%%%%%%%%%% User stories y tareas - Fin %%%%%%%%%%


\begin{document}


%%%%%%%%%%%%%%%%%%%%%%%%%%%%%%%%%%%%%%%%%%%%%%%%%%%%%%%%%%%%%%%%%%%%%%%%%%%%%%%
%% Carátula                                                                  %%
%%%%%%%%%%%%%%%%%%%%%%%%%%%%%%%%%%%%%%%%%%%%%%%%%%%%%%%%%%%%%%%%%%%%%%%%%%%%%%%


\thispagestyle{caratula}

\begin{center}

\includegraphics[height=2cm]{DC.png} 
\hfill
\includegraphics[height=2cm]{UBA.jpg} 

\vspace{2cm}

Departamento de Computación,\\
Facultad de Ciencias Exactas y Naturales,\\
Universidad de Buenos Aires

\vspace{4cm}

\begin{Huge}
\titulo
\end{Huge}

\vspace{0.5cm}

\begin{Large}
\materia
\end{Large}

\vspace{1cm}

\cuatrimestre

\vspace{4cm}

\begin{tabular}{|c|c|c|}
\hline
Apellido y Nombre & LU & E-mail\\
\hline
Izcovich, Sabrina      & 550/11 & sizcovich@gmail.com\\
Lovisolo, Leandro      & 645/11 & leandro@leandro.me\\
Petaccio, Lautaro José & 443/11 & lausuper@gmail.com\\
Vita, Sebastián        & 149/11 & sebastian\_vita@yahoo.com.ar\\
\hline
\end{tabular}

\end{center}

\newpage

\tableofcontents

\newpage


%%%%%%%%%%%%%%%%%%%%%%%%%%%%%%%%%%%%%%%%%%%%%%%%%%%%%%%%%%%%%%%%%%%%%%%%%%%%%%%
%% Introducción                                                              %%
%%%%%%%%%%%%%%%%%%%%%%%%%%%%%%%%%%%%%%%%%%%%%%%%%%%%%%%%%%%%%%%%%%%%%%%%%%%%%%%

\section{Introducción}

En este trabajo se desarrolla una aplicación de envío de avisos, recordatorios y otros tipos de mensajes por medio de SMS a los padres y alumnos de un colegio primario.

La aplicación se especifica y estima utilizando la metodología ágil Scrum, se realiza un diseño bajo el paradigma orientado a objetos respetando un conjunto de reglas de buenas prácticas de diseño y, finalmente, se la implementa en el lenguaje Python.

Para más información, referirse al enunciado de este trabajo práctico\footnote{http://cor.to/enunciadoISW2}.

\subsection{Roles de usuario}

Por brevedad, se utilizan los siguientes roles en las user stories a lo largo de este trabajo:

\begin{description}
  \item[Director] para referirse al director, vicedirector y otros miembros de la dirección del colegio.

  \item[Secretario] para referirse a los miembros de la secretaría del colegio.

  \item[Maestro] para referirse a cualquier maestro del colegio.
\end{description}


%%%%%%%%%%%%%%%%%%%%%%%%%%%%%%%%%%%%%%%%%%%%%%%%%%%%%%%%%%%%%%%%%%%%%%%%%%%%%%%
%% Product backlog                                                           %%
%%%%%%%%%%%%%%%%%%%%%%%%%%%%%%%%%%%%%%%%%%%%%%%%%%%%%%%%%%%%%%%%%%%%%%%%%%%%%%%

\section{Product backlog}

Presentamos, a continuación, las user stories en el product backlog, agrupadas según el tipo de tarea que describen.


\subsection{Gestión de alumnos, padres, cursos y maestros}

\begin{stories}
  \story{director}
        {ingresar los datos de un alumno}
        {inscribirlo en el colegio}

  \story{director}
        {ver los datos de un alumno}
        {actualizarlos o dar de baja al alumno}

  \story{director}
        {ingresar los datos de un padre}
        {darlo de alta en el sistema}

  \story{director}
        {ver los datos de un padre}
        {actualizarlos o dar de baja al padre}

  \story{director}
        {seleccionar un padre}
        {asignarlo como padre de un alumno inscripto}

  \story{director}
        {ingresar el grado, división, turno y aula de un curso}
        {darlo de alta en el sistema}

  \story{director}
        {ver los datos de un curso}
        {actualizarlos o dar de baja el curso}

  \story{director}
        {ingresar los datos de un maestro}
        {darlo de alta en el sistema}

  \story{director}
        {ver los datos de un maestro}
        {actualizarlos o dar de baja al maestro}

  \story{director}
        {seleccionar un maestro}
        {asignárselo a un curso}

  \story{director}
        {seleccionar un conjunto de alumnos}
        {asignarles un curso}
\end{stories}


\subsection{Gestión de eventos}

\begin{stories}
  \story{director, secretario o maestro}
        {ingresar la fecha y descripción de un nuevo evento}
        {darlo de alta en el sistema}

  \story{director, secretario o maestro}
        {ver los datos de un evento}
        {modificarlos o dar de baja el evento}
\end{stories}


\subsection{Gestión de campañas}

\begin{stories}
  \story{director, secretario o maestro}
        {ingresar el título de una nueva campaña y seleccionar un evento existente}
        {crear una campaña asociada al evento seleccionado}

  \story{maestro}
        {seleccionar un subconjunto de mis alumnos}
        {asignarlos como destinatarios de una campaña creada por mí}

  \story{director o secretario}
        {seleccionar un subconjunto de los alumnos del colegio}
        {asignarlos como destinatarios de una campaña creada por mí}

  \story{director, secretario o maestro}
        {ingresar un mensaje, fecha, hora y tipo}
        {agregar un nuevo mensaje a una campaña creada por mí}

  \story{director, secretario o maestro}
        {ver los mensajes de una campaña creada por mí}
        {modificarlos o eliminarlos}

  \story{director, secretario o maestro}
        {que se envíen automáticamente y en el momento correcto los mensajes de una campaña creada por mí}
        {incrementar la eficacia de un evento}

  \story{director, secretario o maestro}
        {ingresar la eficacia de una campaña luego de su terminación}
        {compararla con otras campañas similares}
\end{stories}


%%%%%%%%%%%%%%%%%%%%%%%%%%%%%%%%%%%%%%%%%%%%%%%%%%%%%%%%%%%%%%%%%%%%%%%%%%%%%%%
%% Sprint backlog                                                            %%
%%%%%%%%%%%%%%%%%%%%%%%%%%%%%%%%%%%%%%%%%%%%%%%%%%%%%%%%%%%%%%%%%%%%%%%%%%%%%%%


\section{Sprint backlog}

Las user stories a continuación se toman del product backlog y corresponden al sprint luego del cual se realiza la demo mencionada en el enunciado.

Recordamos los requerimientos de la demo: \emph{``Se presentará una demo requiriendo cierta funcionalidad pedida desde los responsables del proyecto. La misma deberá incluir la posibilidad de definir una campaña para un cierto evento, simular su seguimiento y envío de mensajes, y su posterior evaluación''}.

\begin{center}
\begin{tabular}{|c|l|c|c|}
\hline
ID & Story points & Business value\\
\hline
1 & 123 & 9\\
2 & 444 & 5\\
3 & 111 & 4\\
4 & 555 & 4\\
5 & 123 & 9\\
6 & 123 & 9\\
7 & 123 & 9\\
8 & 123 & 9\\
\hline
\end{tabular}
\end{center}


\subsection{Descripción de user stories, criterios de aceptación y tareas}

\begin{stories}

  \storyid{1}
          {director, secretario o maestro}
          {ingresar la fecha y descripción de un nuevo evento}
          {darlo de alta en el sistema}
        
   \begin{criterios}
    \criteria{La fecha ingresada es correcta.}
  	\criteria{La fecha ingresada es mayor a la fecha actual.}
    \criteria{No se puede validar el evento nuevo si el campo fecha no se encuentra configurado.}
  \end{criterios} 
   
  \begin{tasks}
    \task{Crear interfaz de creación de eventos}
    \task{Crear tabla de eventos en la base de datos}
    \task{Crear código de persistencia de nuevos eventos}
  \end{tasks}


  \storyid{2}
          {director, secretario o maestro}
          {ingresar el título de una nueva campaña y seleccionar un evento existente}
          {crear una campaña asociada al evento seleccionado}
  
   \begin{criterios}
  	\criteria{El título de la campaña tiene más de tres caracteres.}
    \criteria{No se puede crear un título ya existente.}
  \end{criterios} 

  \begin{tasks}
    \task{Crear interfaz de creación de campaña}
    \task{Crear tabla de campañas en la base de datos}
    \task{Crear código de persistencia de nuevas campañas}
  \end{tasks}


  \storyid{3}
          {director o secretario}
          {seleccionar un subconjunto de los alumnos del colegio}
          {asignarlos como destinatarios de una campaña creada por mí}
        
   \begin{criterios}
  	\criteria{El subconjunto de alumnos seleccionado no puede ser vacío.}
    \criteria{El subconjunto de alumnos se asocia a una campaña existente.}
  \end{criterios} 

  \begin{tasks}
    \task{Agregar componente en el formulario de creación de campaña para seleccionar los destinatarios de la misma}
    \task{Crear tablas de relación entre las campañas y los distintos tipos de destinatarios}
    \task{Crear código de persistencia para la lista de destinatarios de la campaña}
  \end{tasks}


  \storyid{4}
          {maestro}
          {seleccionar un subconjunto de mis alumnos}
          {asignarlos como destinatarios de una campaña creada por mí}
            
   \begin{criterios}
  	\criteria{El conjunto de alumnos de mi clase es visible por mí.}
    \criteria{El subconjunto de alumnos seleccionado no es vacío.}
    \criteria{El subconjunto de alumnos se asigna a una campaña existente.}
  \end{criterios} 
   

  \begin{tasks}
    \task{Modificar componente de selección de destinatarios para sólo permitir seleccionar alumnos del maestro y/o sus padres cuando el usuario actual tenga el rol de maestro}
  \end{tasks}


  \storyid{5}{director, secretario o maestro}
             {ingresar un mensaje, fecha, hora y tipo}
             {agregar un nuevo mensaje a una campaña creada por mí}

   \begin{criterios}
  	\criteria{El mensaje nuevo es no vacío.}
    \criteria{La fecha del mensaje es válida.}
    \criteria{La hora del mensaje es válida.}
    \criteria{La hora del mensaje es distinta de la del resto de los mensajes de la campaña de la misma fecha.}
    \criteria{El tipo de mensaje es de alguno de los cuatro tipos permitidos.}
    \criteria{No se puede aprobar la creación del nuevo mensaje si no se validaron la fecha, el tipo, el mensaje y la hora.}
  \end{criterios} 
   
  \begin{tasks}
    \task{Agregar componente tabular en el formulario de creación de campaña que permita ingresar uno a uno los mensajes y visualizar los ya ingresados}
    \task{Crear tabla de mensajes en la base de datos}
    \task{Crear codigo de persistencia de los mensajes ingresados}
  \end{tasks}


  \storyid{6}
          {director, secretario o maestro}
          {ver los mensajes de una campaña creada por mí}
          {modificarlos o eliminarlos}

   \begin{criterios}
    \criteria{El conjunto de todas mis campañas debe ser visible.}
    \criteria{El mensaje modificado pertenece a una de mis campañas.}
  	\criteria{El mensaje modificado no puede ser vacío.}
    \criteria{La hora del mensaje modificado es válida.}
    \criteria{La hora del mensaje modificado es distinta a la del resto de los mensajes de la campaña de la misma fecha.}
    \criteria{El tipo de mensaje modificado es de alguno de los cuatro tipos permitidos.}
    \criteria{El mensaje eliminado debe pertenecer a una de mis campañas.}
  \end{criterios} 

  \begin{tasks}
    \task{Crear una interfaz que liste todas las campañas creadas por el usuario actual}
    \task{Crear interacción que muestre el formulario de edición de la campaña cuando se la selecciona en la lista anterior}
    \task{Modificar formulario de edición de campaña y sus componentes para que tenga precargados los datos de una campaña existente y permita modificarla}
    \task{Modificar código de persistencia de campañas para que permita guardar cambios en una campaña existente}
  \end{tasks}


  \storyid{7}{director, secretario o maestro}
             {que se envíen automáticamente y en el momento correcto los mensajes de una campaña creada por mí}
             {que sean recibidos en el momento adecuado por sus destinatarios}
  
   \begin{criterios}
  	\criteria{El mensaje debe salir a la hora estipulada en la base de datos.}
    \criteria{El mensaje debe ser enviado a todos sus destinatarios previamente configurados.}
  \end{criterios} 

  \begin{tasks}
    \task{Crear una API de envío de SMS que simule el envío guardando los mensajes enviados en una tabla en la base de datos}
    \task{Crear un servicio que periódicamente envíe los mensajes cuyo horario fue alcanzado utilizando el API de envío de SMS}
  \end{tasks}


  \storyid{8}{director, secretario o maestro}
             {ingresar la eficacia de una campaña luego de su terminación}
             {compararla con otras campañas similares}
        
   \begin{criterios}
  	\criteria{El criterio de eficacia en lenguaje natural debe permitir generar un valor de eficacia.}
    \criteria{El sistema debe poder calcular un valor de eficacia a partir del criterio de eficacia en lenguaje natural.}
    \criteria{El valor de eficacia debe ser positivo.}
    \criteria{El criterio de eficacia ingresado en formato numérico debe ser positivo.}
  \end{criterios} 

  \begin{tasks}
    \task{Agregar un campo textual en el formulario de edición de campaña que permita especificar un criterio de eficacia en lenguaje natural}
    \task{Agregar un campo numérico en el formulario de edición de campaña que permita ingresar el valor de eficacia medido}
    \task{Agregar campos en la tabla de campaña para almacenar el criterio de eficacia y la eficacia medida}
    \task{Modificar código de persistencia de campaña para guardar el criterio de eficacia y la eficacia medida}
  \end{tasks}
\end{stories}

\end{document}